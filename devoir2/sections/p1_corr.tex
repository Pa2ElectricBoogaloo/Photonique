{\footnotesize
Soit une cavité électromagnétique avec un mode de fréquence $\omega_c$ contenant un atome. On place l'énergie de l'état fondamental $\ket{g}$ de l'atome à $0$ et on le traite comme un système à deux niveaux dont le niveau excité $\ket{e}$ a une énergie $\hbar \omega_0$. En parallèle, on décrit la cavité avec l'état de Fock $\ket{n}$ associé à un nombre d'occupations $n$ du mode bien défini. Un état complet de la cavité et de l'atome est donné par les produits tensoriels $\ket{g} \otimes \ket{n}$ et le hamiltonien de l'atome est donné par $H_a = \hbar \omega_0\sigma^\dagger \sigma$ où $\sigma$ est l'opérateur d'échelle  défini par $\sigma \ket{e} = \ket{g}$. En ce qui concerne le champ électromagnétique de la cavité, on a le hamiltonien $H_c = \hbar \omega_c a^\dagger a$ où $a$ est l'opérateur d'échelle des occupations de la cavité définie par 
\begin{align}
a = \sum_{k=0}^{\infty} \sqrt{k+1}|k\rangle\langle k+1|. \label{a}
\end{align}

Ensuite, dans le modèle de Jaynes-Cummings on décrit le couplage entre la cavité et l'atome avec un hamiltonien d'interaction $H_{ac} = \hbar g (\sigma^\dagger a + \sigma a^\dagger)$ avec $g$ une constante de couplage. Le hamiltonien total est donné par 
\begin{align}
    H = \hbar \omega_0\sigma^\dagger \sigma + \hbar \omega_c a^\dagger a + \hbar g (\sigma^\dagger a + \sigma a^\dagger). \label{Jaynes}
\end{align}
\begin{enumerate}
    \item En appliquant \eqref{a} sur l'état de Fock $\ket{n}$ on trouve
    \begin{align}
      a \ket{n} = \sum_{k=0}^{\infty} \sqrt{k+1}|k\rangle\bra{k+1}\ket{n} = \sum_{k'=1}^{\infty} \sqrt{k'}|k'-1\rangle \delta_{k', n} = \begin{cases}
         \sqrt{n}|n-1\rangle, \quad n\ge 1,\\ 
         0, \quad n = 0
      \end{cases} 
      = \sqrt{n}|n-1\rangle \label{aa}
    \end{align}
    L'action de l'opérateur conjugué $a^\dagger$ sur $\ket{n}$ produit le résultat 
    \begin{align}
        a \ket{n} = \sum_{k=0}^{\infty} \sqrt{k+1}|k+1\rangle\bra{k}\ket{n} = \sum_{k=0}^{\infty} \sqrt{k}|k+1\rangle \delta_{k, n} =
           \sqrt{n+1}|n+1\rangle \label{aad}
    \end{align}
    \item On considère maintenant l'état général
    \begin{align}
        |\Psi\rangle=\sum_{k\in \mathbb{N}} (c_{e, k}|e, k\rangle+c_{g, k}|g, k\rangle) \label{psi}
    \end{align}
    avec $c_{e, k}, \ c_{g, k} \in \mathbb{C}$. L'évolution temporelle de \eqref{psi} est donné par le Hamiltonien \eqref{Jaynes} combiné à l'équation de Schrodinger qui s'écrit 
    \begin{align}
       0 &=  i \hbar \partial_t |\Psi\rangle - H |\Psi\rangle \nonumber\\
       &= \sum_{k\in \mathbb{N}} (i\hbar\partial_t{c}_{e, k}|e, k\rangle+i\hbar\partial_t{c}_{g, k}|g, k\rangle) - \sum_{k\in \mathbb{N}} \left(\hbar \omega_0\sigma^\dagger \sigma + \hbar \omega_c a^\dagger a + \hbar g  (\sigma^\dagger a + \sigma a^\dagger) \right)(c_{e, k}|e, k\rangle+c_{g, k}|g, k\rangle) \nonumber\\
       &=  \sum_{k\in \mathbb{N}} \left(i\hbar\partial_t{c}_{e, k}|e, k\rangle+i\hbar\partial_t{c}_{g, k}|g, k\rangle - \hbar \omega_0 c_{e, k}|e, k\rangle - \hbar \omega_c k (c_{e, k}|e, k\rangle+c_{g, k}|g, k\rangle)  - \hbar g  (c_{g, k}\sqrt{k}|e, k-1\rangle +  c_{e, k}\sqrt{k+1}|g, k+1\rangle) \right).\label{psi2}
    \end{align}
    Pour simplifier \eqref{psi2}, on effectue les changements d'indices suivants: 
    \begin{align}
        &\sum_{k'=0}^\infty c_{g, k'}\sqrt{k'}|e, k'-1\rangle = \sum_{k = -1}^\infty c_{g, k+1}\sqrt{k+1}|e, k\rangle = \sum_{k\in \mathbb{N}} c_{g, k+1}\sqrt{k+1}|e, k\rangle, \quad k'-1 = k \nonumber\\ 
        &\sum_{k'=0}^\infty  c_{e, k'}\sqrt{k'+1}|g, k'+1\rangle = \sum_{k=1}^\infty  c_{e, k-1}\sqrt{k}|g, k\rangle = \sum_{k\in \mathbb{M}}  c_{e, k-1}\sqrt{k}|g, k\rangle, \quad k'+1 = k \label{expr}
    \end{align}
    où on a autorisé un abus de notation laissant l'indice d'occupation du mode prendre des valeurs négatives pour simplifier la notation (le terme mal défini est multiplié par $0$). En injectant les expressions \eqref{expr} dans \eqref{psi2}, on trouve 
    \begin{align}
        0
        &=  \sum_{k\in \mathbb{N}} \left(i\partial_t{c}_{e, k}|e, k\rangle+i\partial_t{c}_{g, k}|g, k\rangle - \omega_0 c_{e, k}|e, k\rangle - \omega_c k (c_{e, k}|e, k\rangle+c_{g, k}|g, k\rangle)  - g  (c_{g, k+1}\sqrt{k+1}|e, k\rangle +  c_{e, k-1}\sqrt{k}|g, k\rangle) \right)\nonumber \\
        &= \sum_{k\in \mathbb{N}} \left(i\partial_t{c}_{e, k}- \omega_0 c_{e, k} - \omega_c k c_{e, k} - g c_{g, k+1}\sqrt{k+1} \right)|e, k\rangle+\sum_{k\in \mathbb{N}} \left(i\partial_t{c}_{g, k} -  - \omega_c k c_{g, k}  - g c_{e, k-1}\sqrt{k} \right)|g, k\rangle
        .\label{psi3}
     \end{align} 
    L'indépendance linéaire des états $\ket{e, k}$ et $\ket{g, k}$ impose que tous les coefficients du développement \eqref{psi3} soient nuls en tout temps. Cela se traduit par le système d'équations 
    \begin{align}
        \partial_t{c}_{e, k} &= - i (\omega_0 + k \omega_c) c_{e, k} - ig \sqrt{k+1}c_{g, k+1},\nonumber\\
        \partial_t{c}_{g, k} &= - i\omega_c k c_{g, k} - ig \sqrt{k} c_{e, k-1} \iff \partial_t{c}_{g, k+1} = - i\omega_c (k+1) c_{g, k+1} - ig \sqrt{k+1} c_{e, k}
        .\label{eqdiff}
     \end{align} 
    \item On peut utiliser le système \eqref{eqdiff} pour décrire l'évolution de l'état initial $\ket{\Psi(t=0)}=\ket{g, 0}$. En posant $k=0$, on a directement l'équation différentielle 
    $\partial_t{c}_{g, 0} = 0$ qui stipule que l'état ne change jamais et reste donc $\ket{g, 0}$ en tout temps. Ce résultat correspond au fait que si ni l'atome ni le mode de la cavité ne contiennent de quanta d'énergie, il est impossible pour le système d'en acquérir ou d'en échanger entre le mode et l'atome. 
    \item Afin de simplifier le système \eqref{eqdiff}, on procède au changement de variable $c_{e, k} =  e^{-i\omega_c (k+1) t} c'_{e, k}$, $c_{g, k} =  e^{-i\omega_c k t} c'_{g, k}$ pour écrire 
    \begin{align}
        &e^{-i\omega_c (k+1) t}\partial_t{c}'_{e, k} - i\omega_c e^{-i\omega_c (k+1) t} (k+1) {c}'_{e, k} = - i (\omega_0 + k \omega_c) e^{-i\omega_c (k+1) t} c'_{e, k} - ig \sqrt{k+1}e^{-i\omega_c (k+1) t} c'_{g, k+1},\nonumber\\
        &\iff \partial_t{c}'_{e, k}  = i\Delta c'_{e, k} - ig \sqrt{k+1} c'_{g, k+1}, \nonumber \\
        &e^{-i\omega_c (k+1) t} \partial_t{c}'_{g, k+1} - i\omega_c e^{-i\omega_c (k+1) t} (k+1)  c'_{g, k} = - i\omega_c (k+1) e^{-i\omega_c (k+1) t} c'_{g, k+1} - ig \sqrt{k+1} e^{-i\omega_c (k+1) t} c'_{e, k}, \nonumber\\
        &\iff \partial_t{c}'_{g, k+1}  = - ig \sqrt{k+1}  c'_{e, k}
        .\label{eqdiff2}
     \end{align} 
     où $\Delta = \omega_c - \omega_0$.
    \item On considère maintenant l'état initial $\ket{g, n+1}$ pour une cavité en résonnance avec l'atome ($\Delta = 0$). Dans ce cas, \eqref{eqdiff2} se réduit au système 
    \begin{align}
         \partial_t{c}'_{e, k}  = - ig \sqrt{k+1} c'_{g, k+1}\quad \& \quad \partial_t{c}'_{g, k+1}  = - ig \sqrt{k+1}  c'_{e, k}
        .\label{eqdiff3}
     \end{align} 
    On remarque que \eqref{eqdiff3} couplent seulement la composante en $\ket{g, n+1}$ à la composante en $\ket{e, n}$. Pour résoudre \eqref{eqdiff3}, on dérive chaque équation par rapport au temps et on substitue l'autre équation pour trouver 
    \begin{align}
        \partial_t^2 {c}'_{e, n}  = -g^2 (n+1)  c'_{e, n} \quad \& \quad \partial_t^2 {c}'_{g, n+1}  = -g^2 (n+1)  c'_{g, n+1} 
       .\label{eqdiff4}
    \end{align} 
    Pour les deux composantes, \eqref{eqdiff4} corresponds à l'équation d'un oscillateur harmonique associé à la fréquence de Rabi $\Omega = g \sqrt{n+1}$. Avec \eqref{eqdiff3}, les conditions initiales $c_{g, n+1}(0) = c'_{g, n+1}(0) = 1$ et $c_{e, n}(0) = c'_{e, n}(0) = 0$ correspondent  aux dérivées initiales $\partial_t c'_{g, n+1}(0) = 0$ et $\partial_t c'_{e, n}(0) = -i\Omega$. La solution générale de \eqref{eqdiff4} s'écrit 
    \begin{align}
        \ket{\Psi} = (A \cos(\Omega t) + B \sin(\Omega t)) e^{-i\omega_c k t}\ket{g, n+1} + (C \cos(\Omega t) + D \sin(\Omega t))e^{-i\omega_c (k+1) t}\ket{e, n} \label{sol} 
    \end{align}
    Les valeurs initiales des composantes imposent $A = 1, C = 0$ et les valeurs initiales de leurs dérivées imposent $B = 0, D = -i$ ce qui nous laisse la solution 
    \begin{align*}
        \ket{\Psi} = \cos(\Omega t) e^{-i\omega_c k t}\ket{g, n+1} - i \sin(\Omega t) e^{-i\omega_c (k+1) t}\ket{e, n} 
    \end{align*}
    Qui est bien normalisée. La probabilité de mesurer l'atome dans son état excité au temps $t$ est donnée par $P_{g, n+1 \to e, n} = |\bra{e, n}\ket{\Psi}|^2 = \sin^2(\Omega t)$ ce qui confirme que $\Omega$ correspond à une fréquence de Rabi.
    \item La fréquence de Rabi pour un système à deux niveaux quantique (atome d'états $\ket{e}, \ket{g}$) forcé par un champ classique $E_0$ est donnée par $\Omega=-2\langle g|\hat{\varepsilon} \cdot \mathbf{d}| e\rangle E_0 / \hbar$ où $\hat{\varepsilon}$ est la polarisation du champ électrique et $\mathbf{d}$ est l'opérateur moment dipolaire de l'atome. Pour comparer cette fréquence avec celle du modèle \eqref{Jaynes}, on utilise l'expression du champ électrique en seconde quantification pour un seul mode occupé. Il est donné par 
    \begin{align}
        \mathbf{E}(\mathbf{r})=i \sqrt{\frac{\hbar \omega_c}{2 \varepsilon_0 L^3}} \hat{\varepsilon}\left(a e^{i \mathbf{k} \cdot \mathbf{r}}-a^{\dagger} e^{-i \mathbf{k} \cdot \mathbf{r}}\right)\label{E}
    \end{align}
    où $L$ est la largeur d'un côté de la cavité et $\mathbf{r}$ est la position de l'atome dans la cavité. Pour simplifier \eqref{E}, on prend $\mathbf{r}= \mathbf{0}$ et on calcul l'équivalent de l'expression de la fréquence de Rabi semi-classique pour une cavité contenant $n$ quantas comme suit:
    \begin{align}
        \Omega_n = -2 i \sqrt{\frac{\omega_c}{2 \varepsilon_0 L^3 \hbar}} \bra{g, n + 1} \mathbf{E}(\mathbf{0}) \cdot \mathbf{d} \ket{e, n} = -2i \sqrt{\frac{\omega_c}{2 \varepsilon_0 L^3 \hbar}} \bra{g} \hat{\varepsilon} \cdot \mathbf{d} \ket{e} \bra{n+1} (a-a^{\dagger}) \ket{n} = 2i \sqrt{\frac{\omega_c}{2 \varepsilon_0 L^3\hbar }} \bra{g} \hat{\varepsilon} \cdot \mathbf{d} \ket{e} \sqrt{n+1} \label{E2}
    \end{align}
    Pour faire correspondre \eqref{E2} au cas semi-classique, on prend $\hat{\varepsilon} = (i\hat{\varepsilon}_z + \hat{\varepsilon}_x)$ et $\mathbf{d}$ orthogonale à $x$ de sorte que 
    \begin{align}
        \Omega_n = -2 \sqrt{\frac{\omega_c}{2 \varepsilon_0 L^3\hbar }} \bra{g} \hat{\varepsilon}_z \cdot \mathbf{d} \ket{e} \sqrt{n+1} \label{E3}.
    \end{align}
    Avec le résultat \eqref{E3}, on remarque que la fréquence de Rabi pour le modèle quantique obtenue en 5. est retrouvée si on fait l'identification $g =-2 \sqrt{\frac{\omega_c}{2 \varepsilon_0 L^3\hbar }} \bra{g} \hat{\varepsilon}_z \cdot \mathbf{d} \ket{e}$. On peut également faire l'identification $E_0 = \sqrt{\frac{\omega_c}{2 \varepsilon_0 L^3\hbar }} \sqrt{n+1}$ qui indique que la dépendance en $n$ de la fréquence de Rabi quantique est une conséquence de la quantification du champ électromagnétique. Plus $n$ est grand, plus l'intensité du champ électrique est grande et classiquement cela se traduit par un grand $E_0$ effectif dans le terme forçant.
    \newpage 
    \item On s'intéresse ici à un atome initialement préparé sans l'état $\ket{\Psi} = (\ket{g} + \ket{e})/\sqrt{2}$ et ensuite introduit dans une cavité initialement dans l'état de Fock $\ket{n}$. L'état du système est alors $\ket{\Psi} = (\ket{g, n} + \ket{e, n})/\sqrt{2}$. Les états propres de l'atome dans la cavité sont 
    \begin{align}
            &|+\rangle_n=\sin (\theta_n)|g, n+1\rangle+\cos(\theta_n)|e, n\rangle, \nonumber \\
            &|-\rangle_n=\cos (\theta_n)|g, n+1\rangle-\sin(\theta_n)|e, n\rangle \label{eigen}
    \end{align}
    avec $\tan(2 \theta_n) = -2 g \sqrt{n+1}/\Delta$ avec des énergies propres respectives $E_n^{\pm}=(n+1) \hbar \omega_c \pm \hbar g \sqrt{n+1}$. On cherche à décomposer $\ket{\Psi}$ dans la base \eqref{eigen} et on remarque que pour des décalages $\Delta \gg g \sqrt{n+1}$, $\tan(2 \theta_n)\approx 0 \iff \theta_n \approx 0$ ce qui se traduit par $\ket{e, n}\approx \ket{+}_n$ et $\ket{g, n} = \ket{-}_{n-1}$. Autrement dit, à fort décalage spectral, la base des états propres coïncide en bonne approximation avec la base des états $\ket{e, n}$ et $\ket{g, n}$. On peut donc écrire l'évolution temporelle de l'état initial comme suit:
    qui entraine les évolutions temporelles
    \begin{align}
        \ket{\Psi}(t) &\approx e^{-i(n+1)\omega_c-i g \sqrt{n+1} t} |+\rangle_n + e^{-in\omega_c+i g \sqrt{n} t}|-\rangle_{n-1}\nonumber \\ &\approx e^{-i(n+1)\omega_c t-i g \sqrt{n+1} t} |e, n\rangle + e^{-in\omega_c t+i g \sqrt{n} t}|g, n\rangle \nonumber\\ 
        &\approx \left(|e\rangle + e^{i\omega_c t+i g \sqrt{n} t+i g \sqrt{n+1} t}|g\rangle\right) \otimes e^{-i(n+1)\omega_c-i g \sqrt{n+1} t} \ket{n}. \label{phase2}
    \end{align}
    L'expression \eqref{phase2} montre que le passage de l'atome dans la cavité introduit une phase relative entre les états $\ket{e}$ et $\ket{g}$ de la superposition initiale. La fréquence à laquelle ce déphasage évolue dans le temps est $\omega_c + g (\sqrt{n}+ \sqrt{n+1})$ et elle encode donc le nombre de photons dans la cavité. Si $g$ est suffisamment petit, le terme $\omega_c$ correspond à un changement de phase très rapide alors que le terme en $g$ est plus lent. Sur des temps caractéristiques de la période associée à $g$ le terme en $\omega_c$ est globalement moyenné.    









    \item Un état cohérent de la cavité est donné par la série 
    \begin{align} 
        |\alpha\rangle=e^{-|\alpha|^2 / 2} \sum_{k \in \mathbb{N}} \frac{\alpha^k}{\sqrt{k !}}|k\rangle. \label{coherent}
    \end{align}
    avec $\alpha \in \mathbb{C}$.
    Considérons la situation où la cavité et l'atome se trouvent respectivement initialement dans l'état \eqref{coherent} et dans l'état $\ket{e}$. Pour obtenir l'évolution temporelle de cet état initial en résonnance, il suffit d'obtenir celle de $\ket{e, n}$ en utilisant la solution générale \eqref{sol}. On a $c_{g, n+1}(0) = c'_{g, n+1}(0) = 0$ et $c_{e, n}(0) = c'_{e, n}(0) = 1$ qui correspondent aux dérivées initiales $\partial_t c'_{g, n+1}(0) = -i\Omega_n \equiv -ig\sqrt{n+1}$ et $\partial_t c'_{e, n}(0) = 0$. Ces conditions initiales restreignent \eqref{sol} à 
    \begin{align} 
        |e, k\rangle(t) = -i\sin(\Omega_n t)e^{-i\omega_c k t}\ket{g, n+1} + \cos(\Omega_n t)e^{-i\omega_c (k+1) t}\ket{e, n} \label{sol2}
    \end{align}
    En remplaçant chaque état $|e, k\rangle$ de dans la décomposition de $\ket{e}\otimes \ket{\alpha}$ donnée par \eqref{coherent} par son évolution temporelle donnée par \eqref{sol2}, on obtient 
    \begin{align}
        |e, \alpha\rangle(t) &= e^{-|\alpha|^2 / 2} \sum_{k \in \mathbb{N}} \frac{\alpha^k}{\sqrt{k !}}|e, k\rangle(t) \nonumber\\ &= e^{-|\alpha|^2 / 2} \sum_{k \in \mathbb{N}} \frac{\alpha^k}{\sqrt{k !}}(-i\sin(\Omega_k t)e^{-i\omega_c k t}\ket{g, k+1} + \cos(\Omega_k t)e^{-i\omega_c (k+1) t}\ket{e, k}) \nonumber \\
        &= e^{-|\alpha|^2 / 2} \sum_{k \in \mathbb{N}} \left(-i\sin(\Omega_{k-1} t)e^{-i\omega_c (k-1) t}\frac{\alpha^{k-1}}{\sqrt{(k-1)!}}\ket{g, k} + \cos(\Omega_k t)e^{-i\omega_c (k+1) t}\frac{\alpha^k}{\sqrt{k !}}\ket{e, k}\right) \label{sol_co}
    \end{align}
    où on a fait le même abus de notation qu'en 2. laissant la somme passer sur un terme où la notation cesse de fonctionner (contenant ici $(-1)!$) qui est cependant annulé par un facteur ($\Omega_{-1} = g\sqrt{-1+1} = 0$).
    L'expression \eqref{sol_co} permet calculer l'inversion de population $W(t)$ de l'atome comme suit:
    \begin{align}
      W(t) &= |\bra{e} \ket{e, \alpha}(t)|^2 - |\bra{g} \ket{e, \alpha}(t)|^2\nonumber \\
      &= e^{-|\alpha|^2} \sum_{k \in \mathbb{N}} \frac{|\alpha|^{2k}}{k !}\cos^2(\Omega_{k} t)  - e^{-|\alpha|^2} \sum_{k=1}^{\infty} \frac{|\alpha|^{2(k-1)}}{(k-1)!}\sin^2(\Omega_{k-1} t) \nonumber\\
      &= e^{-|\alpha|^2} \sum_{k \in \mathbb{N}} \frac{|\alpha|^{2k}}{k !}(\cos^2(\Omega_{k} t)   - \sin^2(\Omega_{k} t))\nonumber\\
      &= e^{-|\alpha|^2} \sum_{k \in \mathbb{N}} \frac{|\alpha|^{2k}}{k !}\cos(2\Omega_{k} t).\label{popinv}
    \end{align}
    
    \item L'inversion de population $W$ présente une caractéristique intéressante: les oscillations de Rabi apparaissent puis disparaissent en alternance. On cherche ici à qualifier cette alternance. Pour ce faire, on note qu'un état cohérent $\ket{\alpha}$ contient un nombre moyen de photons $\bar{n} = \langle a^\dagger a\rangle = |\alpha|^2$ avec un écart type $  \langle \Delta a^\dagger a\rangle = |\alpha| = \sqrt{\bar{n}}$. Sur la base de \eqref{popinv}, on remarque que les différents états de Fock se combinent avec des fréquences différentes $2 \Omega_n$ et peuvent possiblement interférer destructivement en moyenne. Pour rendre cette remarque plus précise, on considère le temps $t_c$ de première disparition des oscillations de rabbi  où les oscillations des états de Fock à $n \sim \bar{n} \pm \sqrt{\bar{n}}$ sont en opposition de phase. Explicitement, pour $\bar{n}\gg 1$, on a  
    \begin{align}
        \pi \sim (2\Omega_{\bar{n} + \sqrt{\bar{n}}} - 2\Omega_{\bar{n} - \sqrt{\bar{n}}}) t_c &\sim \left(2g\sqrt{\bar{n} + \sqrt{\bar{n}}} - 2g\sqrt{\bar{n} - \sqrt{\bar{n}}}\right) t_c \nonumber \\
        &\sim \sqrt{\bar{n}}\left(2g\sqrt{1 + \sqrt{\bar{n}}/\bar{n}} - 2g\sqrt{1 - \sqrt{\bar{n}}/\bar{n}}\right) t_c \nonumber \\ &\sim \sqrt{\bar{n}}\left(2 + \sqrt{\bar{n}}/\bar{n} - 2 + \sqrt{\bar{n}}/\bar{n}\right) gt_c  = 2g t_c \iff t_c \sim \dfrac{\pi}{2g}. \label{tc}
    \end{align}
    En parallèle, on peut définir le temps $t_r$ de réapparition comme le premier temps où les états de Fock à $n \sim \bar{n}$ et $n \sim \bar{n}+1$ sont en phase. Pour $\bar{n}\gg 1$, on trouve 
    \begin{align*}
        2\pi \sim (2\Omega_{\bar{n} + 1} - 2\Omega_{\bar{n}}) t_c &\sim \left(2g\sqrt{\bar{n} + 1} - 2g\sqrt{\bar{n}}\right) t_c \nonumber \\
        &\sim \left(2g\sqrt{\bar{n}}\sqrt{1 + 1/\bar{n}} - 2g\sqrt{\bar{n}}\right) t_c \nonumber \\ &\sim \left(2\sqrt{\bar{n}} + \sqrt{\bar{n}}/\bar{n} - 2 \sqrt{\bar{n}}\right) gt_c  = \dfrac{g t_c}{\sqrt{\bar{n}}} \iff t_c \sim \dfrac{2\pi \sqrt{\bar{n}}}{g}.
    \end{align*}
    \item Pour une fréquence $\Omega_{\bar{n}}$ fixe, on peut contrôler le temps $t_c$ auquel les oscillations de Rabi disparaissant  de sorte que les oscillations se prolongent à $t \to \infty$. Sur la base de \eqref{tc}, on remarque que les oscillations disparaissent à un temps de plus en plus grand lorsque $g$ devient petit. Pour garder $\Omega_{\bar{n}}$ fixe tout en diminuant $g$ il faut augmenter $\bar{n}$. Ainsi dans la limite à grand $\bar{n}$ et petit $g$ associé à un $\Omega_{\bar{n}}$ fixe, on retrouve des oscillations de Rabi similaires à celles obtenues pour un champ classique. Cela s'explique par le fait qu'à faible $g$, le couplage entre le champ électromagnétique et le système à deux niveaux est suffisamment faible et $\bar{n}$ suffisamment grand pour que le nombre de quantas de champ électromagnétique impliqué ne change pas significativement (de la même manière qu'un champ classique qu'on prend comme une source imperturbable). 
\end{enumerate}
}